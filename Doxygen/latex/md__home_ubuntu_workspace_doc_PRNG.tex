The P\-R\-N\-G test performs a variety of tests on the stream of bytes and produces output as follows on the standard output stream\-: \begin{DoxyVerb}Entropy:
========
Entropy = 8.000000 bits per byte.

Optimum compression would reduce the size
of this 536870912 byte input by 0 percent.

Chi-square Test:
================
Chi square distribution for 536870912 samples is 264.59, and randomly
would exceed this value 32.68 percent of the times.

Arithmetic Mean:
================
Arithmetic mean value of data bytes is 127.4962 (127.5 = random).

Monte Carlo Value for Pi:
=========================
Monte Carlo value for Pi is 3.141875994 (error 0.01 percent).

Serial Correlation Coefficient:
===============================
Serial correlation coefficient is 0.000044 (totally uncorrelated = 0.0).
\end{DoxyVerb}


The values calculated are as follows\-:

\subsection*{Entropy}

The information density of the contents of the file, expressed as a number of bits per character. The results above, which resulted from processing an image file compressed with J\-P\-E\-G, indicate that the file is extremely dense in information—essentially random. Hence, compression of the file is unlikely to reduce its size. By contrast, the C source code of the program has entropy of about 4.\-9 bits per character, indicating that optimal compression of the file would reduce its size by 38\%. \mbox{[}Hamming, pp. 104–108\mbox{]}

\subsection*{Chi-\/square Test}

The chi-\/square test is the most commonly used test for the randomness of data, and is extremely sensitive to errors in pseudorandom sequence generators. The chi-\/square distribution is calculated for the stream of bytes in the file and expressed as an absolute number and a percentage which indicates how frequently a truly random sequence would exceed the value calculated. We interpret the percentage as the degree to which the sequence tested is suspected of being non-\/random. If the percentage is greater than 99\% or less than 1\%, the sequence is almost certainly not random. If the percentage is between 99\% and 95\% or between 1\% and 5\%, the sequence is suspect. Percentages between 90\% and 95\% and 5\% and 10\% indicate the sequence is “almost suspect”. Note that our J\-P\-E\-G file, while very dense in information, is far from random as revealed by the chi-\/square test.

Applying this test to the output of various pseudorandom sequence generators is interesting. The low-\/order 8 bits returned by the standard Unix rand() function, for example, yields\-: \begin{DoxyVerb}Chi square distribution for 500000 samples is 0.01, and randomly
would exceed this value more than 99.99 percent of the times.
\end{DoxyVerb}


While an improved generator \mbox{[}Park \& Miller\mbox{]} reports\-: \begin{DoxyVerb}Chi square distribution for 500000 samples is 212.53, and randomly
would exceed this value 97.53 percent of the times.
\end{DoxyVerb}


Thus, the standard Unix generator (or at least the low-\/order bytes it returns) is unacceptably non-\/random, while the improved generator is much better but still sufficiently non-\/random to cause concern for demanding applications. Contrast both of these software generators with the chi-\/square result of a genuine random sequence created by timing \href{http://www.fourmilab.ch/hotbits/}{\tt radioactive decay} events. \begin{DoxyVerb}Chi square distribution for 500000 samples is 249.51, and randomly
would exceed this value 40.98 percent of the times.
\end{DoxyVerb}


See \mbox{[}Knuth, pp. 35–40\mbox{]} for more information on the chi-\/square test. An interactive \href{http://www.fourmilab.ch/rpkp/experiments/analysis/chiCalc.html}{\tt chi-\/square calculator} is available at this site.

\subsection*{Arithmetic Mean}

This is simply the result of summing the all the bytes (bits if the -\/b option is specified) in the file and dividing by the file length. If the data are close to random, this should be about 127.\-5 (0.\-5 for -\/b option output). If the mean departs from this value, the values are consistently high or low.

\subsection*{Monte Carlo Value for Pi}

Each successive sequence of six bytes is used as 24 bit X and Y co-\/ordinates within a square. If the distance of the randomly-\/generated point is less than the radius of a circle inscribed within the square, the six-\/byte sequence is considered a “hit”. The percentage of hits can be used to calculate the value of Pi. For very large streams (this approximation converges very slowly), the value will approach the correct value of Pi if the sequence is close to random. A 500000 byte file created by radioactive decay yielded\-: \begin{DoxyVerb}Monte Carlo value for Pi is 3.143580574 (error 0.06 percent).
\end{DoxyVerb}


\subsection*{Serial Correlation Coefficient}

This quantity measures the extent to which each byte in the file depends upon the previous byte. For random sequences, this value (which can be positive or negative) will, of course, be close to zero. A non-\/random byte stream such as a C program will yield a serial correlation coefficient on the order of 0.\-5. Wildly predictable data such as uncompressed bitmaps will exhibit serial correlation coefficients approaching 1. See \mbox{[}Knuth, pp. 64–65\mbox{]} for more details. 