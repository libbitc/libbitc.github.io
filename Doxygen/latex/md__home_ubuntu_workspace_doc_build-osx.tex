Instructions by $<$a href=\char`\"{}http\-://github.\-com/colindean\char`\"{}$>$colindean.

\subsection*{Dependencies }

This guide assumes usage of \href{http://brew.sh}{\tt Homebrew} or \href{https://www.macports.org}{\tt Mac\-Ports} for installing dependencies.

You will need to install {\ttfamily G\-M\-P} in order to build {\itshape libbitc}.

Install these packages. It will take a few minutes. \begin{DoxyVerb}brew install autoconf automake libtool argp-standalone gmp
\end{DoxyVerb}


or \begin{DoxyVerb}sudo port install autoconf automake libtool argp-standalone pkgconfig gmp
\end{DoxyVerb}


\subsection*{Building }

Homebrew \begin{DoxyVerb}./autogen.sh
./configure
make
\end{DoxyVerb}


Mac\-Ports \begin{DoxyVerb}./autogen.sh
./configure CPPFLAGS="-I /opt/local/include -L /opt/local/lib"
make
\end{DoxyVerb}


You should also run {\ttfamily make check} in order to run tests. This is a vital step early in the development of {\ttfamily libbitc}.

You can install it if you want with {\ttfamily make install}. It will be installed to {\ttfamily /usr/local/libbitc}.

The {\ttfamily bitsy} binary will be in {\ttfamily ./src}.

\subsection*{Running }

To ensure that at least the basics compiled correctly, execute a command\-: \begin{DoxyVerb}src/bitsy list-settings
\end{DoxyVerb}


You should see output formatted in J\-S\-O\-N, \begin{DoxyVerb}{
  "wallet": "bitsy.wallet",
  "chain": "bitcoin",
  "net.connect.timeout": "11",
  "peers": "bitsy.peers",
  "blkdb": "bitsy.blkdb"
}
\end{DoxyVerb}


If that works, {\ttfamily bitsy} is ready for use. 